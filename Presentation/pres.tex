% Copyright 2004 by Till Tantau <tantau@users.sourceforge.net>.
%
% In principle, this file can be redistributed and/or modified under
% the terms of the GNU Public License, version 2.
%
% However, this file is supposed to be a template to be modified
% for your own needs. For this reason, if you use this file as a
% template and not specifically distribute it as part of a another
% package/program, I grant the extra permission to freely copy and
% modify this file as you see fit and even to delete this copyright
% notice. 



\documentclass{beamer}



\usepackage{ragged2e} 

\usepackage{biblatex}
\addbibresource{TermProject.bib}

%%% Библиография %%%
\makeatletter
\bibliographystyle{utf8gost705u}	% Оформляем библиографию в соответствии с ГОСТ 7.0.5



%%% Поля и разметка страницы %%%
\usepackage{lscape}		% Для включения альбомных страниц
\usepackage{geometry}	% Для последующего задания полей

%%% Кодировки и шрифты %%%
\usepackage{cmap}						% Улучшенный поиск русских слов в полученном pdf-файле
\usepackage[T2A]{fontenc}				% Поддержка русских букв
\usepackage{fontenc}				    % Поддержка русских букв
\usepackage[utf8]{inputenc}				% Кодировка utf8
\usepackage[english, russian]{babel}	% Языки: русский, английский

%%% Математические пакеты %%%
\usepackage{amsthm,amsfonts,amsmath,amssymb,amscd} % Математические дополнения от AMS

%%% Оформление абзацев %%%
\usepackage{indentfirst} % Красная строка

%%% Цвета %%%
\usepackage[usenames]{color}
\usepackage{color}
\usepackage{colortbl}

%%% Таблицы %%%
\usepackage{longtable}					% Длинные таблицы
\usepackage{multirow,makecell,array}	% Улучшенное форматирование таблиц

%%% Общее форматирование
\usepackage[singlelinecheck=off,center]{caption}	% Многострочные подписи
\usepackage{soul}									% Поддержка переносоустойчивых подчёркиваний и зачёркиваний

\setbeamercolor{footline}{fg=grey}
\setbeamertemplate{footline}{
 \leavevmode%
 \hbox{%
 \begin{beamercolorbox}[wd=.333333\paperwidth,ht=2.25ex,dp=1ex,center]{}%
 Стеценко М.А., НИУ ВШЭ
 \end{beamercolorbox}%
 \begin{beamercolorbox}[wd=.333333\paperwidth,ht=2.25ex,dp=1ex,center]{}%
 Москва, 2015
 \end{beamercolorbox}%
 \begin{beamercolorbox}[wd=.333333\paperwidth,ht=2.25ex,dp=1ex,right]{}%
 Стр. \insertframenumber{} из \inserttotalframenumber \hspace*{2ex}
 \end{beamercolorbox}}%
 \vskip0pt%
}

\usetheme{default}

\title{Исследование возможности встраивания контекстной информации в алгоритмы коллаборативной фильтрации на основе матричных разложений}

% A subtitle is optional and this may be deleted
% \subtitle{Optional Subtitle}

\author{Стеценко М. А. \and Игнатов Д. И.}
% - Give the names in the same order as the appear in the paper.
% - Use the \inst{?} command only if the authors have different
%   affiliation.

\institute[Национальный исследовательский университет «Высшая школа экономики»] % (optional, but mostly needed)
{
  Национальный исследовательский университет «Высшая школа экономики» \\
  Факультет компьютерных наук \\ 
  Отделение Прикладной математики и информатики
}
% - Use the \inst command only if there are several affiliations.
% - Keep it simple, no one is interested in your street address.

\date{Москва, 2015}
% - Either use conference name or its abbreviation.
% - Not really informative to the audience, more for people (including
%   yourself) who are reading the slides online

\subject{Theoretical Computer Science}
% This is only inserted into the PDF information catalog. Can be left
% out. 

% If you have a file called "university-logo-filename.xxx", where xxx
% is a graphic format that can be processed by latex or pdflatex,
% resp., then you can add a logo as follows:

% \pgfdeclareimage[height=0.5cm]{university-logo}{university-logo-filename}
% \logo{\pgfuseimage{university-logo}}

% Delete this, if you do not want the table of contents to pop up at
% the beginning of each subsection:

\addtobeamertemplate{block begin}{}{\justifying}

% Let's get started
\begin{document}

\begin{frame}
  \titlepage
\end{frame}

\begin{frame}{Оглавление}
  \tableofcontents
  % You might wish to add the option [pausesections]
\end{frame}

% Section and subsections will appear in the presentation overview
% and table of contents.
\section{Аннотация}

\begin{frame}{Аннотация}
    \begin{block}{}
    Наличие контекстной информации является одним из важнейших факторов для постро- ения личных рекомендаций. Однако, классические алгоритмы коллаборативной фильтрации, основанные на матричных разложениях, таких как SVD разложение, используют только информацию о пользователях и предметах и не предоставляют явных методов включения дополнительных факторов. В данной работе будет показан один из методов встраивания контекстной информации в алгоритм, использующий SVD разложение. Для тестирования рассматривае- мого метода будет использоваться открытый банк данных MovieLens. База данных содержит пользователей портала MovieLens, каждый из которых оценил не менее 20 фильмов, а так же информацию о каждом фильме.
    \end{block}
\end{frame}

\begin{frame}{Аннотация}
    \begin{block}{}
    Context has always been an important factor in personalized Recommender systems. However, standard collaborative filtering algorithms based on matrix factorization rely mainly on user and subject information and don’t provide any methods for incapsulating extra data. This work demonstrates such a method based on SVD decomposition. To test results an open data base taken from MovieLens is used. The database provides information about users and movies.
    \end{block}
\end{frame}

\section{Введение}

% You can reveal the parts of a slide one at a time
% with the \pause command:
\begin{frame}{Введение}
  \begin{itemize}
  
  \item {
    \textbf{Рекомендательные системы} – это модели, которые лучше вас знают, чего вам хочется.
    \pause
    
    \begin{itemize}
    
    \item {
        Netflix - аренда фильмов.
    }
    
    \pause
    
    \item {
        Amazon - лидер в области E-Commerce.
    }
    
    \pause
    
    \item {
        Яндекс.Музыка - стриминг музыки.
    }
    
    \pause
    
    \end{itemize}
    
  }
  
  \item {   
    Для множества пользователей User и предметов Objects составляем матрицу, каждый элемент матрицы $a_{i,j}$ хранит оценку пользователя i для предмета j. Оценки может и не быть, зачастую такие матрицы очень разрежены, много нулей. Теперь задача заключается в заполнении пропусков.
    \pause
  }
 
  \item {
    \textbf{Коллаборативная фильтрация} - метод построения рекомендаций, основываясь на оценках других пользователей.
  }
  
  \end{itemize}
\end{frame}


\begin{frame}{Введение}

    \begin{itemize}
    
    \item {
        Множество переменных, которые влияют на отношение пользователя к предмету и следовательно на рекомендации для этого пользователя, называется \textbf{контекстом}. Например, жанр фильма, время покупки товара, темп музыки.
    
    }
    
    \item {
        \textbf{Цель работы}: научиться использовать контекстные данные для улучшения результатов работы рекомендательной системы.
    }
    
    \end{itemize}

\end{frame}

% Placing a * after \section means it will not show in the
% outline or table of contents.
\section{Заключение}

\begin{frame}{Заключение}
  \begin{itemize}
  \item {
    Изучение известных матричных разложений (Singular Value Decomposition, Non-negative matrix factorization).
  }
  
  \item {
    Поиск вариантов встраивания контекстной информации в эти методы.
  }
  
  \item {
    Реализация системы на языке Python.
  }

  \item {
    Анализ эффективность в сравнении с обычными моделями, не использующие контекст.
  }
  
  \end{itemize}

    \nocite{*}
\end{frame}

\section{Литература}

\begin{frame}[t,allowframebreaks]
  \frametitle{Литература}
  \printbibliography
 \end{frame}


\end{document}


