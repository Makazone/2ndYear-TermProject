\begin{abstract}

%%% По-русски %%%
\textbf{"Исследование возможности встраивания контекстной информации в алгоритмы коллаборативной фильтрации на основе матричных разложений"}\par
\textbf{Стеценко Макар Александрович}\par
Факультет компьютерных наук. Второй курс. 203 группа. 2015 год.\par
\textbf{Игнатов Дмитрий Игоревич}\par
Факультет компьютерных наук. Департамент анализа данных и искусственного интеллекта.\par
Научный руководитель. dignatov@hse.ru\par

\vspace{5mm}

Наличие контекстной информации является одним из важнейших факторов для построения личных рекомендаций. Однако, классические алгоритмы коллаборативной фильтрации, основанные на матричных разложениях, таких как SVD разложение, используют только информацию о пользователях и предметах и не предоставляют явного методов включения дополнительных факторов. В данной работе будет показан один из методов встраивания контекстной информации в алгоритм, использующий SVD разложение. Для тестирования рассматриваемого метода будет использоваться открытый банк данных \href{http://grouplens.org/datasets/movielens}{MovieLens}. База данных содержит пользователей портала MovieLens, каждый из которых оценил не менее 20 фильмов, а так же информацию о каждом фильме.  

\end{abstract}

\clearpage

\renewcommand{\abstractname}{Abstract}
\begin{abstract}
%%% In English %%%
\par\textbf{"Integrating Contextual Information into Collaborative Filtering Algorithms based on Matrix Decomposition"}\par
\textbf{Stetsenko Makar}\par
Faculty of Computer Science. 2nd course. 203 group. 2015 year.\par
\textbf{Ignatov Dmitry}\par
Faculty of Computer Science. School of Data Analysis and Artificial Intelligence.\par
Scientific adviser. dignatov@hse.ru\par

\vspace{5mm}

Context has always been an important factor in personalized Recommender systems. However, standard collaborative filtering algorithms based on matrix factorization rely mainly on user and subject information and don't provide any methods for incapsulating extra data. This work demonstrates such a method based on SVD decomposition. To test results an open data base taken from \href{http://grouplens.org/datasets/movielens}{MovieLens} is used. The database provides information about users and movies.   

\end{abstract}