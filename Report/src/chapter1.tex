\newchapter{1}{Глава 1}

\section{Введение}

Рекомандательные системы стали назависимой областью для исследования примерно в середине 90-ых. Сегодня можно дать такое определение: Рекомендательные системы - это программы, которые помогают пользователю принять решение, стараясь найти из общего множества товаров, набор товаров наиболее схожий с уже понравившимся ему товарами. Что считать понравившимся товаром, зависит от области, в которой будет применяться рекомендательная система, это может быть оценка фильму или переход по ссылке на страницу продукта. Поиск таких товаров имеет первостепенную роль для онлайн бизнеса. Интернет магазины и сервисы по предоставляю контента имеют в своем распоряжении тысячи наименований, но лишь малая доля будет интересна конечному пользователю, поэтому построение правильных рекомендаций гарантированно увеличивает прибыль таких сервисов. В качестве примера можно привести интернет гиганта в сфере E-Commerce - Amazon, а так же Netflix, компанию, специализирующуюся на показе фильмов и сериалов.   

\section{Типа рекомендательных систем}

Существует два основных подхода для нахождения рекомендаций: \textit{Content-based Filtering} и \textit{Collaborative Filtering (CF)}.

\subsection{Content-based Filtering}

Данный метод использует свойства и содержание рекомендуемых объектов. В системе основанной на данном подходе выделяют 3 основных компонента:

\begin{itemize}
\item \textbf{Парсер}. Этот компонент нужен для выделения нужной информации из объекта и ее структурирования. В качестве объекта может выступать веб-страница, набор действий пользователя, текстовый документ и так далее. Используются различные методы \textit{feature extraction} для выделения ключевых свойств объекта, например, текстовый документ можно представить вектором ключевых слов.

\item \textbf{Конструктор пользовательского профиля}. Этот модуль получает уже структурированные данные и пытается на их основе построить профиль пользователя. 

\item \textbf{Фильтр}. Данный компонент отвечает за построение конкретных рекомендаций. Имея готовый профиль пользователя и некоторую метрику, позволяющую измерить сходство между двумя объектами, строится ранжированный список наиболее похожих предметов, которые и являются результатом работы всей системы.

\end{itemize}

\subsection{Collaborative Filtering}

Данный метод основывается на большом количестве собранных от пользователей отзывов, а не свойствах рекомендуемого объекта. Главной целью является поиск схожих пользователей, основываясь на оценках, которые они поставили предмету. Найдя такие группы, довольно легко построить рекомендации.

В данной работе будет исследоваться коллаборативная фильтрация, поэтому формализуем решаемую ей задачу, она так же называется \textit{задачей о рекомендациях}.

Прежде чем ввести формальное определение задачи, обозначим множество пользователей за \mathcal{U}, множество предметов за \mathcal{I} и множество всех оценок за \mathcal{R}. Так же оговорим, что никакой пользователь не оценивал один и тот же предмет дважды, тогда оценка пользователя $u$ предмету $i$ запишем, как $r_{ui}$. Можно составить матрицу, где по строкам будут пользователи, по столбцам предметы, а элемент матрицы будет из множества оценок \mathcal{R}. В среднем каждый пользователь оценивает не больше 10-20 предметов, поэтому в матрице будет очень много элеметов для которых оценка $r_{ui}$ неизвестена. Задача рекомендательной системы предсказать значение $r_{ui}$.

\section{Способы реализации CF}

Существует два основных метода решения задачи о рекомендациях используя CF.

\begin{itemize}
\item Метод соседей (Neighborhood based) 
\item Матричное разложение
\end{itemize}

\subsection{Метод соседей}

Данный метод очень простой в реализации и основывается на поиске множества пользователей, похожих на пользователя $u$. Так как каждый пользователь представляет собой вектор оценок, то схожесть двух пользователей можно оценить, посчитав косинус угла между двумя векторами. После того, как были найдены $n$ схожих пользователей, легко пресказать непроставленные оценки пользователя $u$, используя имеющиеся оценки найденных пользователей.

\subsection{Матричное разложение}

Поскольку матрицу оценок пользователей может быть очень большой, а заполненных ячеек в ней очень мало, можно уменьшить размер пространства путем поиска некоторого общего набора факторов $f_i$ которые будут общими, как для пользователей, так и для предметов. Смысл матричной факторизации заключается в приближении исходной большой матрицы произведением нескольких, но меньших по размеру.

В данной работе будет рассмотрено наиболее популярное и применяемое разложение - SVD (Singular Value Decomposition).

$$
M_{m,n} = U_{m,f} x K_{f,f} x I_{n,f}
$$

Здесь матрица слева - это исходная матрица оценок (m = $|mathcal{U}|$, n = $|mathcal{I}|$. Матрицы справа и есть искомое разложение, рассмотрим его подробнее. Элемент матрицы $U$ содержит веса каждого из факторов для конкретного пользователя, другими словами, элемент $u_{i,j}$ говорит насколько важен пользователя $\mathcal{U_i}$ фактор $f_j$. Тоже самое, но для предметов, содержит матрица $I$. Матрица $K$ диагональная, на ее диагонали в убывающем порядке, находятся сингулярные числа матрицы $M$, они показыывают, насколько фактор $f_i$ важен, в контексте рекоментальных систем, он показывает какую долю оценки занимает фактор $f_i$, например, жанр фильма может иметь большое значение для пользователя, в то время как год выпуска - нет. Сами факторы $f$ не имеют никакой интерпритации, и SVD разложение не говорит, что это за факторы. Найдя такое разложение, не только уменьшается объем памяти, затрачиваемый на работу рекомендательной системы, но и легко решается задача о рекомендациях. Перемножив эти три матрицы, получится матрица исходная матрица оценок, но она уже будет полной, а не разреженной.

В данной работе, будет использоваться именно SVD разложение.