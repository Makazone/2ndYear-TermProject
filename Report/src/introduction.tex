\newchapter{1}{Введение}

\setcounter{page}{5}

Растущее число различных заболеваний, как врожденных, так и приобретенных, вынуждает искать новые методы лечения. Воздействие напрямую на гены - один из таких методов. Например, зная ген, который отвечает за иммунитет, можно активизировать его, тем самым "включив" иммунитет людям, у которых он отсутствует при рождении. Или, к примеру, воздействуя на определенные гены в раковой опухоли, можно подавить её, излечив человека.

\textbf{Целью} данной работы является создание динамической модели поведения генов внутри живой клетки. Это позволит наблюдать, что произойдет с определенными генами (активизируются они или, наоборот, будут подавлены) при воздействии различными препаратами или терапиями, и позволит в какой-то мере прогнозировать ход лечения.

Для достижения поставленной цели необходимо было решить следующие задачи:
\begin{enumerate}
  \item По книгам \cite{bib1, bib2} изучить язык программирования JavaScript.
  \item Изучить основные принципы построения веб-приложений.
  \item Изучить JavaScript библиотеку для визуализации данных D3.js
  \item Изучить литературу по биологии клетки и генетике \cite{bib3, bib4, bib5}.
  \item Разобраться, в каком формате выгоднее подавать граф программе
  \item Написать эту динамическую модель
  \item Добавить возможность загрузки произвольного графа в нужном формате
  \item Добавить возможность приостанавливать процесс в любой момент и изменять параметры генов
\end{enumerate}

\textbf{Научная новизна:}
\begin{enumerate}
  \item Разработанная специально для этого эмулятора модель сигнального каскада.
  \item Впервые создан открытый и свободный эмулятор, позволяющий любому человеку провести исследование.
\end{enumerate}

\textbf{Научная и практическая значимость} данной работы состоит в том, что созданная динамическая модель позволяет в короткие сроки проводить исследование набора генов (например, генов раковой опухоли) и выявлять гены, воздействие на которые, позволит излечить больного.

\clearpage

\textbf{План работы:} 
\begin{table} [htbp]
  \centering
  \parbox{15cm}{\caption{План работы}\label{Ts0Sib}}
%  \begin{center}
  \begin{tabular}{| p{3cm} || p{12cm}l |}
  \hline
  \hline
  Дата & Работа, выполненная к этой дате & \\
  \hline
  \hline
  31.01.2015 & Закончить изучение JavaScript по книгам \cite{bib1, bib2} & \\
  \hline
  15.02.2015  & Изучить основные принципы построения веб-приложений & \\
  \hline
  22.02.2015 & Разобраться с D3.js & \\
  \hline
  15.03.2015 & Закончить изучение литературы по биологии клеток и генетике \cite{bib3, bib4, bib5} & \\
  \hline
  22.03.2015 & Определиться с форматом хранения графов & \\
  \hline
  05.04.2015 & Написать первую рабочую версию & \\
  \hline
  12.04.2015 & Добавить возможность загрузки произвольного графа в нужном формате & \\
  \hline
  19.04.2015 & Добавить возможность приостанавливать процесс в любой момент и изменять параметры генов & \\
  \hline
  \hline
  \end{tabular}
%  \end{center}
\end{table}

\clearpage