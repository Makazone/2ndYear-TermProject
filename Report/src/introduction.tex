\newchapter{1}{Введение}

\setcounter{page}{5}

Рекомендательные системы стали неотъемлемой частью жизни людей и бизнеса, особенно после популяризации BigData и открытых данных. Задача таких систем угадывать предпочтения пользователей основываясь на их предыдущих действиях. Более формально, задача о рекомендациях состоит в угадывании значения пары (пользователь, предмет). В простейшем варианте, ответом на данный вопрос будет либо 1 - предмет интересен пользователю, либо 0 - предмет не стоит рекомендовать. \textit{Коллаборативная фильтрация} один из способов построения рекомендаций. Данный метод использует информацию о других пользователях системы, чтобы понять, какой предмет скорее всего понравится рассматриваемому пользователю. Несмотря на то, что упрощенная модель \textit{(пользователь, предмет)} подходит для моделирования многих ситуаций, очень часто приходится сталкиваться с дополнительными параметрами, которые играют важную роль при решении задачи о рекомендациях. Таким параметром может быть время, тогда уже имеется 3-х мерный вектор \textit{(пользователь, предмет, время}. Множество переменных, которые влияют на отношение пользователя к предмету и следовательно на рекомендации для этого пользователя, называется \textit{контекстом}. 

\textbf{Целью} данной работы будет решение задачи о рекомендациях с использованием контекстной информации.

\clearpage

\textbf{План работы:} 
\begin{table} [htbp]
  \centering
  \parbox{15cm}{\caption{План работы}\label{Ts0Sib}}
%  \begin{center}
  \begin{tabular}{| p{3cm} || p{12cm}l |}
  \hline
  \hline
  Дата & Цель & \\
  \hline
  \hline
  31.01.2015 & Подробнее ознакомиться с задачей о рекомендациях \cite{IgnatovFCA, IgnatovWitology, XavierAmatriain, Alqadah2014} & \\
  \hline
  15.02.2015 & Ознакомиться с языком программирования Java \cite{EckelJava} & \\
  \hline
  22.03.2015 & Написать первый вариант программы & \\
  \hline
  05.04.2015 & Посчитать быстродействие и точность на различных наборах данных & \\
  \hline
  12.04.2015 & Исправление ошибок в работе программы, улучшение & \\
  \hline
  19.04.2015 & Закончить работу над отчетом & \\
  \hline
  \hline
  \end{tabular}
%  \end{center}
\end{table}

\clearpage